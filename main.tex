\documentclass[runningheads,a4paper]{llncs}

% Umlaute unter UTF8 nutzen
\usepackage[utf8]{inputenc}

% Variablen
% Folgende Variablen stehen in der ganzaen Arbeit zur Verfügung.
\newcommand{\artDerArbeit}{Projektarbeit/Bachelorarbeit/Masterarbeit}
\newcommand{\titelDerArbeit}{Titel}
\newcommand{\autorDerArbeit}{Max Mustermann}
\newcommand{\institutHinterDerArbeit}{Fachhochschule Dortmund}
\newcommand{\mailAutorDerArbeit}{max.musterman042@stud.fh-dortmund.de}

% weitere Pakete
% Grafiken aus PNG Dateien einbinden
\usepackage{graphicx}

% Deutsche Sonderzeichen und Silbentrennung nutzen
\usepackage[ngerman]{babel}

% Eurozeichen einbinden https://ctan.org/pkg/eurosym
\usepackage[right]{eurosym}

% Zeichenencoding
\usepackage[T1]{fontenc}

% https://tex.stackexchange.com/questions/147194/is-it-still-useful-to-load-the-lmodern-package
\usepackage{lmodern}

% mehrseitige Tabellen ermöglichen https://tex.stackexchange.com/questions/469892/how-to-use-longtable-in-latex
\usepackage{longtable}

% Unterstützung für Schriftarten
%\newcommand{\changefont}[3]{ 
%\fontfamily{#1} \fontseries{#2} \fontshape{#3} \selectfont}

% bricht lange URLs "schön" um
\usepackage[hyphens,obeyspaces,spaces]{url}

% Paket für Textfarben
\usepackage{color}

% Mathematische Symbole importieren
\usepackage{amssymb}

% Mathematische Symbole importieren
\usepackage{amssymb}

%\usepackage[bookmarksnumbered,pdftitle={\titleDocument},hyperfootnotes=false]{hyperref}

\usepackage[
    backend=biber,
    style=alphabetic, % https://www.overleaf.com/learn/latex/Biblatex_citation_styles
    sortlocale=de_DE,
    natbib=true,
    url=false,
    doi=true,
    eprint=false
]{biblatex}
\addbibresource{bibliography}

% für Bildbezeichner
\usepackage{capt-of}

% für Stichwortverzeichnis
\usepackage{makeidx}

% für Listings
\usepackage{listings}
\lstset{numbers=left, numberstyle=\tiny, numbersep=5pt, keywordstyle=\color{black}\bfseries, stringstyle=\ttfamily,showstringspaces=false,basicstyle=\footnotesize,captionpos=b}
\lstset{language=java}

% Indexerstellung
\makeindex

% Abkürzungsverzeichnis
\usepackage[german]{nomencl}
\let\abbrev\nomenclature
% Abkürzungsverzeichnis LiveTex Version
% Titel des Abkürzungsverzeichnisses
\renewcommand{\nomname}{Abkürzungsverzeichnis}
% Abstand zwischen Abkürzung und Erläuterung
\setlength{\nomlabelwidth}{.25\textwidth}
% Zwischenraum zwischen Abkürzung und Erläuterung mit Punkten
\renewcommand{\nomlabel}[1]{#1 \dotfill}
% Variation des Abstandes der einzelnen Abkürzungen zu einander
\setlength{\nomitemsep}{-\parsep}
% Index mit Abkürzungen erzeugen
\makenomenclature
%\makeglossary

\begin{document}
    % hier werden die Trennvorschläge inkludiert
    \hyphenation{
Film-pro-du-zen-ten
Lux-em-burg
Soft-ware-bau-steins
zeit-in-ten-siv
}

    
    % lncs-Umgebung aus Variablen setzen 
    \title{\titelDerArbeit}
    \author{\autorDerArbeit}
    \institute{\institutHinterDerArbeit}
    \email{\mailAutorDerArbeit}
    \maketitle
    \input{inhalte/00_abstract.tex}

    % Inhaltsverzeichnis anzeigen
    \newpage
    \tableofcontents

    % das Abbildungsverzeichnis
    %\newpage
    % Abbildungsverzeichnis soll im Inhaltsverzeichnis auftauchen
    \addcontentsline{toc}{section}{Abbildungsverzeichnis}
    % Verion 1: Abbildungsverzeichnis MIT führender Nummberierung endgueltig anzeigen
    \listoffigures

    % das Tabellenverzeichnis
    %\newpage
    % Tabellenverzeichnis soll im Inhaltsverzeichnis auftauchen
    \addcontentsline{toc}{section}{Tabellenverzeichnis}
    % \fancyhead[L]{Abbildungsverzeichnis / Abkürzungsverzeichnis} %Kopfzeile links
    % Tabellenverzeichnis endgültig anzeigen
    \listoftables
    
    % das Abkürzungsverzeichnis
    %\newpage
    % Abkürzungsverzeichnis soll im Inhaltsverzeichnis auftauchen
    \addcontentsline{toc}{section}{Abkürzungsverzeichnis}
    % das Abkürzungsverzeichnis entgültige Ausgeben
    %\fancyhead[L]{Abkürzungsverzeichnis} %Kopfzeile links
    %\nomenclature{UGC}{User Generated Content}

    \printnomenclature[3cm]

    % hier der eigentliche Input
    \section{Einleitung}

\subsection{Motivation}

\subsection{Zielsetzung}

\subsection{Vorgehensweise}



    % Literaturliste soll im Inhaltsverzeichnis auftauchen
    \newpage
    \addcontentsline{toc}{section}{Literaturverzeichnis}
    % Literaturverzeichnis anzeigen
    \renewcommand\refname{Literaturverzeichnis}
    \printbibliography


\end{document}